\section{Conclusion et pistes de recherche futures}
    Toutes les expérimentations ont permis de valider l'utilisation d'un auto-encodeur dans un système d'attention si certaines contraintes sont respectées. De plus, il a été possible de déterminer les problèmes de différents modèles, de déterminer le meilleur modèle et d'invalider un modèle. Pour compléter l'étude actuelle, il serait intéressant de réaliser les éléments suivants.

    \begin{itemize}
        \item Tester avec d'autres \textit{backend} pour l'extraction des caractéristiques dans le but d'améliorer les performances;
        \item Ajouter un terme à la fonction de coût d'entraînement sur le tenseur des caractéristiques des modèles C et D  pour permet l'entraînement du réseau d'extraction des caractéristiques dans le but d'éviter la détérioration des performances de validation au fil de l'entraînement;
        \item Tester les modèles sur d'autres bases de données pour s'assurer de la validité de nos conclusions dans d'autres environnements;
        \item Prendre en compte l'aspect temporel des images des vidéos;
        \item Faire des tests supplémentaires pour mesurer l'effet de l'augmentation des données de contexte
        \item Tester avec un robot possédant un mécanisme d'attention comme celui de HBBA;
        \item Rendre la méthode d'annotation plus systématique pour l'établissement du critère de performance.
    \end{itemize}
    
    Ces éléments pourraient permettre d'améliorer les performances des modèles et de généraliser les conclusions de cette étude.